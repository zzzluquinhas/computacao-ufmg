\documentclass[12pt]{article}
\usepackage[utf8]{inputenc}
\usepackage[brazilian]{babel}
\usepackage{hyperref}
\usepackage{lipsum}
\usepackage{geometry}
\usepackage[skip=5pt plus1pt, indent=20pt]{parskip}
\usepackage{indentfirst}
\usepackage{graphicx}
\usepackage[center]{caption}
\usepackage[font=small]{caption}
\usepackage{subcaption}
\usepackage{placeins}
\usepackage{setspace}
\usepackage{amsthm,amssymb,amsmath}
\usepackage{algorithm}
\usepackage{algpseudocode}
\usepackage[hashEnumerators,smartEllipses]{markdown}

\renewcommand*\familydefault{\sfdefault} 

\graphicspath{{./midia/}}

\geometry{margin=2cm}

\title{\textbf{Trabalho Prático 3 - Centro de Distribuição}}
\author{\textbf{Lucas Almeida Santos de Souza - 2021092563\textsuperscript{1}}}
\date{\parbox{\linewidth}{\centering%
	\textsuperscript{1}Universidade Federal de Minas Gerais (UFMG)\endgraf
	Belo Horizonte - MG - Brasil\endgraf\bigskip
	\href{mailto:luscaxalmeidass@ufmg.br}{luscaxalmeidass@ufmg.br}}}

\begin{document}

\maketitle

%%%%%%%%%%%%%%%%%%%%%%%%%%%%%%%%%%%%%%%%%%%%%%%%%%%%%%%%%%%%%%%%%%%%%%%%%%%%%%

\section{Introdução}

	\par Este trabalho consiste na resolução de um problema utilizando programação dinâmica. Apresentaremos o problema, sua modelagem, a solução proposta e a análise de complexidade. Por fim, será apresentada uma argumentação para provar que se trata de um problema NP-Completo.

	\subsection*{Descrição do problema}

		\par O problema em questão envolve um centro de distribuição de ligas metálicas que possui conexões com várias fábricas e clientes, e tem o objetivo de otimizar sua logística. Cada fábrica produz ligas metálicas de tamanhos diferentes, e cada cliente possui uma demanda específica de ligas, medida em metros. O objetivo é minimizar o número de ligas necessárias para atender à demanda, levando em consideração os tamanhos disponíveis nas fábricas.

		\par Por exemplo, se o centro de distribuição tem ligas de tamanhos [1, 5, 10, 20, 25] e um cliente precisa de 40 metros de ligas, o número mínimo de ligas necessárias para suprir essa demanda é 2 (duas ligas de 20 metros cada).

\section{Modelagem}

	\par A entrada do programa consiste em um arquivo de texto com o seguinte formato:

	\begin{itemize}
		\item A primeira linha contém um inteiro $T$ que representa o número de testes que será feito;
		\item A segunda linha contém dois inteiros separados por espaço:
		\begin{itemize}
			\item O primeiro, $1 \leq N \leq 1000$, representa a quantidade de tamanhos de ligas metálicas disponíveis (ou o número de fábricas);
			\item O segundo, $1 \leq W \leq 1000000$, representa a demanda do cliente em metros.
		\end{itemize}
		
		\item A terceira linha contém uma sequência de inteiros $1 \leq l_i \leq 1000$ separados por espaço, que representam os tamanhos de ligas disponíveis, em metros.
	\end{itemize}

	\par A saída do programa consiste em um arquivo de texto com $T$ linhas, cada uma contendo um inteiro que representa o número mínimo de ligas necessárias para atender à demanda de cada teste.

	\subsection*{Implementação}
	\label{sec:implementacao}

		\par O pseudocódigo da função que resolve o problema pode ser visto abaixo:
		
		\vspace{12pt}
		\hrule
		\vspace{3pt}
		\hrule
		\noindent\textbf{minimo(vector $l$, int $w$):} \\
		\noindent\begin{tabular}{l}
			\texttt{resultados := [$\infty$]} \footnotesize \textit{// Vetor de tamanho $w + 1$ inicializado em $\infty$} \\
			\indent \indent \indent \indent \footnotesize \textit{// contém o resultado para cada tamanho de encomenda} \\
			\\
			\texttt{resultados[0] := 0} \footnotesize \textit{// Não é necessário nenhuma liga para uma encomenda de tamanho 0} \\
			\\
			\textbf{para cada} tamanho de liga $l_i$ \textbf{faça} \\
			\indent \textbf{para cada} tamanho de encomenda j=1 até w \textbf{faça} \\
			\indent \indent \textbf{se} $l_i$ $\leq$ j \textbf{faça} \\
			\indent \indent \indent \texttt{resultados[j] = min(resultados[j], resultados[j - $l_i$] + 1)} \\
			\indent \indent \textbf{fim se} \\
			\indent \textbf{fim para cada} \\
			\textbf{fim para cada} \\
			\\
			\textbf{retorne} \texttt{resultados[w]}
		\end{tabular}
		\hrule
		\vspace{3pt}
		\hrule
		\vspace{12pt}

		\par O algoritmo apresentado acima utiliza conceitos de programação dinâmica para resolver o problema de modo mais eficiente. O vetor \texttt{resultados}, inicializado em $\infty$, contém o número mínimo de ligas necessárias para atender à demanda de cada tamanho de encomenda. Para cada tamanho de liga disponível menor ou igual à demanda, o algoritmo calcula o número mínimo de ligas necessárias para atender à demanda de tamanho $j$ utilizando a liga de tamanho $l_i$. O resultado é o mínimo entre o número de ligas necessárias para atender à demanda de tamanho $j$ utilizando a liga de tamanho $l_i$ e o número mínimo de ligas necessárias para atender à demanda de tamanho $j - l_i$ (já calculado anteriormente) mais uma liga de tamanho $l_i$.

\section{Análise de complexidade e NP-Completude}
	\subsection*{Análise de complexidade}
		\par Poderia se pensar inicialmente que o algoritmo apresentado possui complexidade $O(NW)$, onde $N$ é o número de tamanhos de ligas disponíveis e $W$ é a demanda do cliente. Isso porque o algoritmo itera sobre cada tamanho de liga disponível e para cada tamanho de liga disponível itera sobre cada tamanho de encomenda de 1 até $W$.
		
		\par No entanto, enquanto o vetor \texttt{resultados} afeta a complexidade em seu tamanho $N$ (independente do tamanho dos números que ele contém), a demanda $W$ do cliente afeta a complexidade em sua grandeza, mesmo sendo um só número. Então, uma maneira mais correta de analisar a complexidade do algoritmo é em função do número de bits necessários para representar $W$ em binário. Considerando $Y$ como o número de bits necessários para representar $W$ em binário, então a complexidade do algoritmo é $O(N \cdot 2^Y)$.

	\subsection*{NP-Completude}
		\par Para provar que o problema é NP-Completo, é necessário reduzir a um problema NP-Completo conhecido. O problema da soma de subconjuntos (Subset Sum) é um problema NP-Completo que consiste em determinar se existe um subconjunto de um dado conjunto cuja soma dos elementos é igual a um determinado valor. Esse problema é bem parecido com o problema das ligas metálicas, mas o problema da soma de soma de conjuntos contabiliza cada valor apenas uma vez, enquanto o problema das ligas metálicas contabiliza cada valor quantas vezes forem necessárias para atender à demanda.

		\par Então, para reduzir o problema das ligas metálicas ao problema da soma de subconjuntos, dado uma entrada $([a_1, a_2, ..., a_n], S)$ basta seguir os seguintes passos para gerar uma entrada $([l_1, ..., l_n, l_1', ..., l_n'], W)$ para o problema das ligas metálicas:

		\par Seja $b = max(n+1, S+1)$, $l_1 = 1$ e para cada $i \in \{1,...,n\}$, faça:
		
		\begin{equation}
			\begin{split}
				l_i = a_i \cdot b^{n+1} + b^i \\
				l_i' = b^i
			\end{split}
		\end{equation}

		\par E, por fim,

		\begin{equation}
			W = S \cdot b^{n+1} + \sum_{i=1}^{n} b^i
		\end{equation}

		\par Ao converter os números do subconjunto orginal para a base b desse modo, o problema das ligas metálicas se torna obrigado a escolher uma liga de cada tipo, pois se escolher duas ligas do mesmo tipo, o custo somado será maior do que escolher uma liga de cada tipo. Além disso, o problema das ligas metálicas se torna obrigado a escolher a liga de menor tamanho de cada tipo, pois se escolher uma liga de tamanho maior, o custo somado será maior do que escolher a liga de menor tamanho.

		\par Após essa função, basta analisar o retorno do problema das ligas metálicas. Se o retorno for $n$, então o problema das ligas metálicas encontrou uma solução que atende à demanda de tamanho $W$ utilizando $n$ ligas, logo existe o subconjunto com soma $k$ do problema da soma de subconjuntos. Se o retorno for $\infty$, então o problema das ligas metálicas não encontrou uma solução que atende à demanda de tamanho $W$, logo não existe o subconjunto com soma $k$. Portanto, o problema das ligas metálicas é NP-Completo.

\end{document}